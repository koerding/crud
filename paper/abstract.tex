\section*{Significance Statement}

In large datasets from biomedicine, neuroscience, psychology, and genomics, nearly every pair of measured features is significantly correlated---the so-called ``crud factor.'' These correlations are real, not artifacts of multiple testing: they reflect shared latent structure distributed broadly across many dimensions. We show that standard statistical adjustments cannot remove this background because shared variance in real-world data follows approximately power-law spectra, so no moderate-rank correction eliminates it. Many small associations reported as statistically significant are indistinguishable from this background, even after controlling for confounders. We prove that when an effect is comparable in size to the domain's background correlation scale, no purely statistical method can reliably separate signal from noise, regardless of sample size. We propose a practical diagnostic---a crud-aware $z$-test---that benchmarks any reported correlation against the domain background. Only associations that clearly exceed this background should be treated as evidence for a direct causal relationship.

\begin{abstract}
Causal claims in observational research increasingly rest on small adjusted correlations in large datasets. We quantify the background distribution of pairwise correlations across nine datasets spanning health, neuroscience, psychology, genomics, politics, and vision. After removing shared variation via principal component regression, the typical scale of residual correlations among random variable pairs---the ``crud scale'' $\sigma_K$---shrinks only slowly with the number of removed components. We trace this to approximately $1/f$ eigenvalue spectra, which distribute shared variance across many directions so that no moderate-dimensional adjustment can eliminate background dependence. A decision-theoretic result formalizes the consequence: when the causal signal is comparable to $\sigma_K$, no association-only rule can reliably distinguish direct causal relations from background dependence, regardless of sample size. We propose a crud-aware $z$-test that benchmarks observed correlations against the domain background. Associations that are highly significant by classical standards can become entirely unremarkable once background dependence is accounted for.
\end{abstract}
