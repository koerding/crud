\section{Introduction}

Across biomedicine, social science, and behavioral research, causal claims increasingly rely on observational associations in large datasets. Randomized experiments remain the benchmark for counterfactual effects, but intervention is often infeasible, so small adjusted associations are routinely interpreted as evidence about causal mechanism. This paper asks how informative those associations remain in the presence of the background dependence structure typical of modern datasets.

Code and the interactive demo are available at \url{https://github.com/koerding/crud} and \url{https://koerding.github.io/crud/}. The draft PDF is at \url{https://raw.githubusercontent.com/koerding/crud/master/paper/crud.pdf}.

We organize interpretation around three uses of correlation: large effects that clear the crud scale easily, prediction without causal claims, and small-effect causal claims that depend on adjusted associations standing out from background dependence. Our results target \emph{only} this third use. We do not argue that correlation is useless---large effects easily clear any background, and prediction tasks do not require causal interpretation. The concern is specifically with small adjusted associations that are interpreted as evidence for direct causal relationships.

Related work emphasizes that widespread weak dependence is common: Meehl's ``crud factor'' \citep{meehl1990} documented that in behavioral data, essentially everything correlates with everything. Sensitivity analyses \citep[e.g.,][]{cinelli2020,oster2019} provide tools to benchmark how strong unobserved confounding would need to be to explain away an observed association. Their question is hypothetical: ``how strong would confounding need to be to nullify this result?'' Ours is empirical: ``what does background dependence actually look like in this domain?'' The two approaches are complementary, not competing. For a broad annotated discussion of the ``everything is correlated'' perspective, see \citet{gwernEverything}. Separately, approximate $1/f$ spectral structure has been documented across many complex systems \citep{press1978,keshner1982,bak1987}. Our contribution is to connect these threads: we quantify the background correlation distribution across domains, link its slow shrinkage under generic adjustment to eigenvalue spectra, and propose a crud-aware calibration for interpreting small adjusted associations.

A concrete example previews the issue. A recent study reported a nominally significant association between folate intake and mortality in adults with type~2 diabetes ($r \approx 0.02$, $p < 0.05$, $n \approx 8{,}000$) \citep{liu2022}. But when we compare this association to the background distribution of correlations among NHANES variables---the ``crud''---it falls squarely in the middle: its crud-aware $p$-value is $0.81$, meaning the association is indistinguishable from a randomly chosen variable pair in the same dataset (Section~\ref{sec:worked-examples}). The classical test detects that the correlation is nonzero; it does not detect that the correlation is distinctive.

By ``meaningfully correlated with causation,'' we mean the following narrow question. After a deliberately generic adjustment that removes broad shared variation (here operationalized via regression on the top $K$ principal component scores), do the remaining pairwise associations separate from the domain's typical residual dependence? Specifically, are they large enough to support stable causal interpretation without additional design leverage (randomization, instruments, discontinuities, negative controls)? We operationalize the background dependence scale as the residual correlation standard deviation across uniformly sampled feature pairs and use it as the benchmark for whether an adjusted association is surprising. A formal bridge from the eigenvalue spectrum to this crud scale, and a decision-theoretic limit for association-only causal decisions, are given in Appendix~\ref{app:bridge-lemmas} (Theorems~\ref{thm:crud-scale}--\ref{thm:assoc-only}). Across the nine domains we study, most small adjusted associations do not separate from the domain's background dependence.

Our approach treats the distribution of correlations among uniformly sampled feature pairs as a diagnostic of the domain's generic dependence structure. Under the common sparsity assumption that direct causal relations are rare relative to the number of measurable variables \citep[cf.][Ch.~2]{spirtes2000}, most random pairs should not be directly connected, so the background correlation distribution is the relevant null for whether an observed association is distinctive. Concretely, we (i) quantify the empirical background correlation distribution across domains, (ii) measure how it changes under rank-$K$ PC regression, (iii) document approximately power-law eigenvalue spectra, and (iv) propose evaluating associations against a crud-aware empirical null. The central contribution is the logic: if crud is ubiquitous and spectra are broad, then small adjusted associations are often not distinctive relative to the domain background; the bridge lemma formalizes this link.
